\documentclass[UTF8]{ctexart}
\usepackage{bm}
\usepackage{amssymb}
\usepackage{mathtools}
\usepackage{amsmath}
\usepackage{float}
\usepackage{rotating}
\usepackage{booktabs}
\usepackage{pdfpages}

\title{\heiti 最优化第六次作业}
\author{\kaishu 张晋15091060}
\begin{document}
\maketitle
\begin{enumerate}
\item[2.32] 原问题系数矩阵如下:

\begin{table}[H]
\centering
	\begin{tabular}{ccccccc}
	\toprule
	{}&$x_1$&$x_2$&$x_3$&$x_4$&$\bm{b}$\\
	\midrule
    {}    & 1     & 2     & 1     & 2     & 3 \\
    {}    & 1     & 1     & 2     & 4     & 5 \\
    $\bm{c}^T$    & 2     & 3     & 2     & 2& 0    \\
	\bottomrule
	\end{tabular}
\end{table}

\begin{table}[H]
\centering
	\begin{tabular}{ccccccc}
	\toprule
	{}&$x_1$&$x_2$&$x_3$&$x_4$&$\bm{B}^{-1}\bm{b}$\\
	\midrule
    {}    & 1     & 2     & 1     & 2     & 3 \\
    {}    & 0     & -1     & 1     & 2     & 2 \\
    $\bm{c}^T$    & 2     & 3     & 2     & 2& 0    \\ 
	\bottomrule
	\end{tabular}
\end{table}

\begin{table}[H]
\centering
	\begin{tabular}{ccccccc}
	\toprule
	{}&$x_1$&$x_2$&$x_3$&$x_4$&$\bm{B}^{-1}\bm{b}$\\
	\midrule
    {}    & \boxed{1}     & 3     & 0    & 0    & 1 \\
    {}    & 0     & -1     & 1     & 2     & 2 \\
  $\bm{c}^T$    & 2     & 3     & 2     & 2& 0    \\ 
	\bottomrule
	\end{tabular}
\end{table}

\begin{table}[H]
\centering
	\begin{tabular}{ccccccc}
	\toprule
	{}&$x_1$&$x_2$&$x_3$&$x_4$&$\bm{B}^{-1}\bm{b}$\\
	\midrule
    {}    & 1     & 3     & 0    & 0    & 1 \\
    {}    & 0     & -1     & \boxed{1}     & 2     & 2 \\
    $\bm{r}^T$    & 0     & -3     & 2     & 2& -2    \\
	\bottomrule
	\end{tabular}
\end{table}

\begin{table}[H]
\centering
	\begin{tabular}{ccccccc}
	\toprule
	{}&$x_1$&$x_2$&$x_3$&$x_4$&$\bm{B}^{-1}\bm{b}$\\
	\midrule
    {}    & 1     & 3     & 0    & 0    & 1 \\
    {}    & 0     & -1     & 1    & \boxed{2}     & 2 \\
    $\bm{r}^T$    & 0     & -1     & 0     & -2& -6 \\
	\bottomrule
	\end{tabular}
\end{table}

\begin{table}[H]
\centering
	\begin{tabular}{ccccccc}
	\toprule
	{}&$x_1$&$x_2$&$x_3$&$x_4$&$\bm{B}^{-1}\bm{b}$\\
	\midrule
    {}    & 1     & \boxed{3}     & 0    & 0    & 1 \\
    {}    & 0     & -1/2     & 1/2    & 1     & 1 \\
    $\bm{r}^T$    & 0     & -2     & 1     & 0& -4 \\
	\bottomrule
	\end{tabular}
\end{table}

\begin{table}[H]
\centering
	\begin{tabular}{ccccccc}
	\toprule
	{}&$x_1$&$x_2$&$x_3$&$x_4$&$\bm{B}^{-1}\bm{b}$\\
	\midrule
       {}    & 1/3   & 1     & 0     & 0     & 1/3 \\
    {}    & 1/6   & 0     & 1/2   & 1     & 7/6 \\
   $\bm{r}^T$    & 2/3   & 0     & 1     & 0     & −10/3 \\
	\bottomrule
	\end{tabular}
\end{table}

故得到该问题的最优解为$\bm{x}=(0,\dfrac{1}{3},0,\dfrac{7}{6})^T$,最优值为$10/3$

\[\bm{B}=
\begin{bmatrix}
2 &2\\
1& 4
\end{bmatrix},\qquad
\bm{B}^{-1}=\dfrac{1}{6}
\begin{bmatrix}
4 &-2\\
-1& 2
\end{bmatrix}\]

\[\bm{x}_B=\bm{B}^{-1}
\begin{bmatrix}
1\\
8
\end{bmatrix}=
\begin{bmatrix}
-2\\
5/2
\end{bmatrix},\qquad
z=\bm{c}_B^T\bm{x}_B=-1\]

可得新单纯形表:
\begin{table}[H]
\centering
	\begin{tabular}{ccccccc}
	\toprule
	{}&$x_1$&$x_2$&$x_3$&$x_4$&$\bm{x}^{B}$\\
	\midrule
       {}    & 1/3   & 1     & 0     & 0     & -2 \\
    {}    & 1/6   & 0     & 1/2   & 1     & 5/2 \\
   $\bm{r}^T$    & 2/3   & 0     & 1     & 0     & 1 \\
	\bottomrule
	\end{tabular}
\end{table}

由于$\dfrac{1}{3}x_1+x_2=-2$没有非负解,故无可行解.

\item[2.34]
\begin{enumerate}
\item 对偶单纯形表如下:

\begin{table}[H]
\centering
	\begin{tabular}{ccccccccc}
	\toprule
	{}&$x_1$&$x_2$&$x_3$&$x_4$&$x_5$&$x_6$&$\bm{x}^{B}$\\
	\midrule
    {}    & -2    & -1    & \boxed{-4}    & 0     & 1     & 0     & -2 \\
   {}   & -2    & -2    & 0     & -4    & 0     & 1     & -3 \\
    $\bm{r}^T$     & 12    & 8     & 16    & 12    & 0     & 0     & 0 \\
	\bottomrule
	\end{tabular}
\end{table}

\begin{table}[H]
\centering
	\begin{tabular}{ccccccccc}
	\toprule
	{}&$x_1$&$x_2$&$x_3$&$x_4$&$x_5$&$x_6$&$\bm{x}^{B}$\\
	\midrule
   {}    & 1/2   & 1/4   & 1     & 0     & -1/4  & 0     & 1/2 \\
    {}    & \boxed{-2}    & -2    & 0     & -4    & 0     & 1     & -3 \\
    $\bm{r}^T$    & 4     & 4     & 0     & 12    & 4     & 0     & -8 \\
	\bottomrule
	\end{tabular}
\end{table}

\begin{table}[H]
\centering
	\begin{tabular}{ccccccccc}
	\toprule
	{}&$x_1$&$x_2$&$x_3$&$x_4$&$x_5$&$x_6$&$\bm{x}^{B}$\\
	\midrule
    {}   & 0     &\boxed{-1/4}  & 1     & -1    & -1/4  & 1/4   & -1/4 \\
    {}    & 1     & 1     & 0     & 2     & 0     & -1/2  & 3/2 \\
      $\bm{r}^T$     & 0     & 0     & 0     & 4     & 4     & 2     & -14 \\
	\bottomrule
	\end{tabular}
\end{table}

\begin{table}[H]
\centering
	\begin{tabular}{ccccccccc}
	\toprule
	{}&$x_1$&$x_2$&$x_3$&$x_4$&$x_5$&$x_6$&$\bm{x}^{B}$\\
	\midrule
     {}    & 0     & 1     & -4    & 4     & 1     & -1    & 1 \\
     {}    & 1     & 0     & 4     & -2    & -1    & 1/2   & 1/2 \\
       $\bm{r}^T$     & 0     & 0     & 0     & 4     & 4     & 2     & -14 \\
	\bottomrule
	\end{tabular}
\end{table}

故得到该问题的最优解为$\bm{x}=(\dfrac{1}{2},1,0,0,0,0)^T$,最优值为$14$.

\item 其对偶问题如下:
\begin{alignat}{2}
max \quad & -2\lambda_1-3\lambda_2 \nonumber\\
\mbox{s.t.}\quad
&-2\lambda_1-2\lambda_2 \leq 12\nonumber\\
&-\lambda_1-2\lambda_2\leq 8\nonumber\\
&-4\lambda_1 \leq 16\nonumber\\
&-4\lambda_2 \leq 12\nonumber\\
&\lambda_1,\lambda_2 \leq 0
\end{alignat}

由图像可知,其最优解为$\bm{\lambda}=(-4,-2)^T$,最优值为$14$.


\item 由$\bm{r}^T_N=\bm{c}^T_N-\bm{\lambda}^T\bm{N}$得:$\qquad\bm{\lambda}=-\bm{r}_N$,观察表中的$(r_5,r_6)$即可得到$\lambda_1=(0,0),\quad \lambda_2=(-4,0),\quad \lambda_1=(-4,-2),\quad \lambda_1=(-4,-2),\quad $
可见单纯形乘子沿着对偶问题的基本可行解移动,最后移动到最优点.

\includepdfmerge{fig.pdf}
\end{enumerate}
\item[2.36]
\item 原问题转化为约束问题:
\begin{alignat}{2}
max \quad & 2x_1+2x_2+3x_3 \nonumber\\
\mbox{s.t.}\quad
&x_1+2x_2+2x_3\leq 12\nonumber\\
&2x_1+4x_2+x_3\leq f\nonumber\\
&x_i \geq 0 \qquad(i=1,2,3)
\end{alignat}

其对偶问题如下:
\begin{alignat}{2}
min \quad &12\lambda_1+f\lambda_2 \nonumber\\
\mbox{s.t.}\quad
&\lambda_1,\lambda_2 \geq 0\nonumber\\
&\lambda_1+2\lambda_2 \geq 2\nonumber\\
&2\lambda_1+4\lambda_2 \geq 2\nonumber\\
&2\lambda_1+\lambda_2\geq 3
\end{alignat}

由图解法可以看出:
\[\bm{\lambda} = \begin{cases}
L_4,& f=0\\
(0,3),& f\in (0,6)\\
L_3,& f=6\\
(4/3,1/3),& f \in (6,24)\\
L_2,&f=24\\
(2,0),& f \in (24,\infty)\\
\lambda_2=\infty,& f<0
\end{cases}\]

即:
\[\lambda_2(f) = \begin{cases}
[3,\infty),& f=0\\
3,& f\in (0,6)\\
[1/3,3],& f=6\\
1/3,& f \in (6,24)\\
[0,1/3],&f=24\\
0,& f \in (24,\infty)\\
\infty,& f<0
\end{cases},\qquad
z(f) = \begin{cases}
3f,& f\in [0,6)\\
f/3+16,& f \in [6,24)\\
24,& f \in [24,\infty)\\
\text{无解},& f<0
\end{cases}\]

\item
设棉花的需求量为$T\geq s$,则
\[\Delta Z=T/6-(T-s)/6-s/12=s/12\]
\[\pi(s) = \begin{cases}
z(f)+s/12,& s \leq T\\
z(f)+T/12,& s > T\\
\end{cases}\]
\includepdfmerge{f6.pdf}
\includepdfmerge{f7.pdf}
\end{enumerate}


\end{document}