\documentclass[UTF8]{ctexart}
\usepackage{bm}
\usepackage{amssymb}
\usepackage{mathtools}
\usepackage{amsmath}
\usepackage{float}
\usepackage{rotating}
\usepackage{booktabs}
\usepackage{pdfpages}

\title{\heiti 最优化第十次作业}
\author{\kaishu 张晋15091060}
\begin{document}
\maketitle
\begin{enumerate}
\item[4.7]
\begin{enumerate}
\item \[\lim_{x\rightarrow \infty}\bm{x}^{(k)}=0,\quad e_k=\dfrac{1}{k},\quad \lim e_{k+1}/e_{k}\rightarrow 1\]
收敛阶为1,为次线性收敛,$k=10001$

\item \[\lim_{x\rightarrow \infty}\bm{x}^{(k)}=0,\quad e_k=(\dfrac{1}{2})^{2^k},\quad \lim e_{k+1}/e_{k}^2\rightarrow 1\]
收敛阶为2,为二次收敛,$k=\text{IntegerPart}\left[\log _2\left(4 \log _2(10)\right)\right]+1=4$

\item \[\lim_{x\rightarrow \infty}\bm{x}^{(k)}=0,\quad e_k=\dfrac{1}{k!},\quad \lim e_{k+1}/e_{k}\rightarrow 0\]
收敛阶趋近于1,为超线性收敛,$k=8$
\end{enumerate}

\item[4.8]
\begin{equation}
\bm{g}(\bm{x})=\nabla f(\bm{x})=(\dfrac{\partial f}{\partial x_1},\dfrac{\partial f}{\partial x_2})^T=[2(x_1+{x_2}^2),4(x_1+{x_2}^2)x_2]^T
\end{equation}

\begin{equation}
\bm{g}(\bm{x}^{(k)})=(2,0)^T
\end{equation}

\begin{equation}
{\bm{p}^{(k)}}^T\bm{g}^{(k)}=-2<0
\end{equation}
故$\bm{p}^{k}$是其下降方向。

由于$f(\bm{x})$并非二次函数,所以使用$\alpha_k=\dfrac{{-{\bm{p}^{(k)}}^T}\bm{g}^{(k)}}{{\bm{p}^{(k)}}^T\bm{G}\bm{p}^{(k)}}$计算$\alpha$时会出现误差,此时应该直接代入算得:

\begin{equation}
f(\bm{x}^{(k+1)})=(\alpha^2-\alpha+1)^2
\end{equation}
故$\alpha=1/2$时,$f(\bm{x}^{(k+1)})$取极小值


\item[4.11]
\begin{equation}
\phi(\alpha)=1-\alpha e^{-\alpha^2},\qquad \phi(0)=1
\end{equation}

\begin{equation}
\phi'(\alpha)=(2\alpha^2-1)e^{-\alpha^2},\qquad \phi'(0)=-1
\end{equation}


\begin{equation}
\phi(\alpha)\leq \phi(0)+\rho \phi'(0) \alpha
\end{equation}


\begin{equation}
\Rightarrow \qquad \alpha \leq \sqrt{-\ln \rho}
\end{equation}

\begin{equation}
\phi(\alpha)\geq \phi(0)+(1-\rho) \phi'(0) \alpha
\end{equation}


\begin{equation}
\Rightarrow \qquad \alpha \geq \sqrt{-\ln (1-\rho)}
\end{equation}

Goldstein条件:
\begin{equation}
\sqrt{-\ln (1-\rho)} \leq \alpha \leq \sqrt{-\ln \rho}
\end{equation}

Wolfe条件:
\begin{equation}
(2\alpha^2-1)e^{-\alpha^2}\geq -\sigma \quad \& \quad \alpha \leq \sqrt{-\ln \rho}
\end{equation}

强Wolfe条件:
\begin{equation}
-\sigma \leq (2\alpha^2-1)e^{-\alpha^2}\leq \sigma \quad \& \quad \alpha \leq \sqrt{-\ln \rho}
\end{equation}

代入$\sigma,\rho$解得:

\begin{table}[H]
\centering
	\begin{tabular}{ccc}
	\toprule
	{}&$\sigma=\rho=1/10$&$\sigma=\rho=1/4$\\
	\midrule
    Goldstein条件   &  $[0.324593,1.517427]$&$[0.536360,1.177410]$\\
    Wolfe条件   &  $[0.650865,1.517427]$&$[0.571578,1.177410]$\\
    强Wolfe条件    &    $[0.650865,0.768257]$&$[0.571578,0.877473]$ \\
	\bottomrule
	\end{tabular}
\end{table}

\end{enumerate}
\end{document}